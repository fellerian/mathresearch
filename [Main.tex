%----------------------------------------------------------------------------------------------------------------
%	P A G E    S E T - U P
%----------------------------------------------------------------------------------------------------------------

%\documentclass{article}
\documentclass[leqno,11pt]{report}
%\documentclass[reqno,11pt]{report}
%\topmargin=-0.45in
%\evensidemargin=0in
%\oddsidemargin=0in
%\textwidth=6.5in
%\textheight=9.0in
%\headsep=0.25in
\usepackage{geometry}
\geometry{bindingoffset=0cm}
\geometry{textwidth=390pt}
\textheight=8.5in
\headsep=0.15in


\pagestyle{empty}
\setlength{\parindent}{5pt}   
% packages for fancy fonts, symbols, thm/proof environments, etc
 

\makeatletter
\newcommand\useleqno{\renewcommand\@eqnnum{\hb@xt@.01\p@{}%
                      \rlap{\normalfont\normalcolor
                        \hskip -\displaywidth(\theequation)}}}
                        
%----------------------------------------------------------------------------------------------------------------
%	F O N T S
%---------------------------------------------------------------------------------------------------------------

%\usepackage[sc]{mathpazo}
%\linespread{1.05}         % Palatino needs more leading (space between lines)
%\usepackage[T1]{fontenc}
%\usepackage{mathpazo}
%\usepackage{tgpagella}
\usepackage[mathscr]{euscript}



%\usepackage{pxfonts}
%\usepackage{quoting}
%\quotingsetup{font={itshape,footnotesize}}



%----------------------------------------------------------------------------------------------------------------
%	P A C K A G E S
%----------------------------------------------------------------------------------------------------------------


%------------------------------------------------------------------------------ C O M M A N D S


\usepackage{amsmath}
\usepackage{amssymb,amsthm, bm}
%\usepackage{esint}
\usepackage{setspace}
\usepackage{nicefrac}
\usepackage{enumitem, kantlipsum}
\usepackage{calc}
\usepackage{soul}
\usepackage{upgreek}

\usepackage{import}
%\usepackage[table]{xcolor}
\usepackage{booktabs}



\usepackage[linktoc=all]{hyperref}
\hypersetup{
    colorlinks,
    citecolor=black,
    filecolor=black,
    linkcolor=black,
    urlcolor=black
}

\setlength{\fboxsep}{0.5em}
\usepackage{empheq}
\usepackage{centernot}
\usepackage{blkarray}

\allowdisplaybreaks

\usepackage{textgreek}
%\usepackage[euler]{textgreek}

%\usepackage[cal=boondox]{mathalfa}
\usepackage{scalerel}
\usepackage{amsfonts} 
%\usepackage{mathabx}
%\usepackage{verbatim}




%------------------------------------------------------------------------------ P A G E   S E T - U P

\usepackage{nicefrac}
\usepackage{fancyhdr} % Required for custom headers
\usepackage{lastpage} % Required to determine the last page for the footer
\usepackage{extramarks} % Required for headers and footers
%\usepackage[usenames,dvipsnames]{color} % Required for custom colors
\usepackage[dvipsnames]{xcolor}

\usepackage{graphicx} % Required to insert images
\graphicspath{{figures}}
\usepackage{listings} % Required for insertion of code
\usepackage{courier} % Required for the courier font
\usepackage{lipsum} % Used for inserting dummy 'Lorem ipsum' text into the template
\usepackage{titlesec}


\usepackage{standalone}
\usepackage{changepage}
\usepackage[makeroom]{cancel}
\usepackage{scalerel,stackengine}
\usepackage{floatrow}
\usepackage{caption}
\usepackage{empheq}
\usepackage{fancyhdr}
\usepackage{mdframed}
\usepackage{bbm}



\usepackage{titlesec}
\usepackage{tikz-cd}
\usepackage{accents}
\usepackage{mathtools}
\usepackage{multirow}
\usepackage{tcolorbox}
\usepackage{subcaption}
%\usepackage{physics}
\usepackage[italicdiff]{physics}
\usepackage{pifont}
\usetikzlibrary{positioning,arrows}

%\usepackage{array}


%----------------------------------------------------------------------------------------------------------------
%	N E W   C O M M A N D S
%----------------------------------------------------------------------------------------------------------------

\newcommand{\into}{\lrcorner \, }
\newcommand{\inn}{\, \in \, }
\newcommand{\mr}{\mathring}
\newcommand{\hs}{\hslash}
\newcommand{\simm}{\ \sim \ }
\newcommand{\frontwc}[1]{\prescript{\vee}{}{#1}}
\newcommand{\bw}{\bigwedge \nolimits}
\newcommand{\bv}{\bigvee \nolimits}
\newcommand{\pseu}{\psi\text{DO}}
\newcommand{\wlim}[1]{\underset{#1}{\nm{weak\,}\lim}\ }
\newcommand{\ccdot}{{\,\bm{\cdot} \,}}
\newcommand{\ctimes}{\! \cdot \!}
\newcommand{\mdot}{{\cdot}}
\newcommand{\Ad}{\nm{Ad}}
\newcommand{\Ric}{\nm{Ric}}
\newcommand{\Alt}{\nm{Alt}}
\newcommand{\para}{/\!/}
\newcommand{\dist}{\nm{dist}}
\newcommand{\csubset}{\subset \subset}
\newcommand{\subsett}{\ \subset \ }
\newcommand{\inclu}{\hookrightarrow}
\newcommand{\kap}{\kappa}
\newcommand{\cis}{\nm{\,cis\,}}
\newcommand{\ldim}{\underline{\nm{dim}}}
\newcommand{\udim}{\overline{\nm{dim}}}
%\newcommand{\sech}{\nm{sech}}
\newcommand{\bs}[1]{\boldsymbol{#1}}
\newcommand{\rcur}{{\scr r}}
\newcommand{\brcur}{\bm \rcur}
\newcommand{\nab}{\nabla}
\newcommand{\ome}{\omega}
\newcommand{\Ome}{\Omega}
\newcommand{\dto}{\downarrow}
\newcommand{\uto}{\uparrow}
\newcommand{\del}{\delta}
\newcommand{\Del}{\Delta}
\newcommand{\tha}{\theta}
\newcommand{\Tha}{\Theta}
\newcommand{\img}{\nm{im\,}}
\newcommand{\at}[1]{\!^{\,\, #1}}
\newcommand{\emp}{\varnothing}
\newcommand{\corr}{\ \longleftrightarrow \ }
\newcommand{\Corr}{\ \Leftrightarrow \ }
\newcommand\finline[3][]{\begin{myfont}[#1]{#2}#3\end{myfont}}%
\newcommand{\sket}{\text{Sketch of Proof}}
\newcommand{\re}{\nm{Re\,}}
\newcommand{\im}{\nm{Im\,}}
\newcommand{\dual}{\nm{dual\,}}
\newcommand{\ol}{\overline}
\newcommand{\sig}{\sigma}
\newcommand{\Sig}{\Sigma}
\newcommand{\alp}{\alpha}
\newcommand{\gam}{\gamma}
\newcommand{\Gam}{\Gamma}
\newcommand{\Div}{\nabla \cdot}
\newcommand{\vDiv}{\vec \nabla \cdot}
\newcommand{\Curl}{\nabla \times}
\newcommand{\Lap}{\Del}
\newcommand{\Grad}{\nab}
\newcommand{\vGrad}{\vec \nab}
\newcommand{\lap}{\bm{\Del}}
\newcommand{\id}{\textnormal{id}}
\newcommand{\emphr}[1]{\emph{\textcolor{red}{#1}}}
\newcommand{\embf}[1]{\emph{\textbf{#1}}}
\newcommand{\emphred}[1]{\emph{\textcolor{red}{#1}}}
\newcommand{\sgn}{\textnormal{sgn}}
\newcommand{\half}{\frac{1}{2}}
\newcommand{\thalf}{\tfrac{1}{2}}
\newcommand{\refl}{\text{ref}}
\newcommand{\Capa}{\textnormal{Cap}}
\newcommand{\capa}{\textnormal{Cap}}
\newcommand{\1}{\mathbbm{1}}
\newcommand{\oX}{\prescript{o}{}X}
\newcommand{\pX}{\prescript{p}{}X}
\newcommand{\Xo}{X_o}
\newcommand{\Xp}{X_p}
\newcommand{\geqs}{\geqslant}
\newcommand{\leqs}{\leqslant}
\newcommand{\wh}{\widehat}
\newcommand{\wt}{\widetilde}
\newcommand{\Dom}{\nm{Dom}\,}
\newcommand{\Prob}{\mathbb{P}}
\newcommand{\Var}{\nm{Var}}
%\newcommand{\var}{\textnormal{var}}
\newcommand{\Cov}{\nm{Cov}}
\newcommand{\cov}{\nml{cov}}
\newcommand{\supp}{\nm{supp}\,}
\newcommand{\ut}[1]{\underset{\wt{}}{#1}}



\newcommand{\benum}{\begin{enumerate}}
\newcommand{\eenum}{\end{enumerate}}
\newcommand{\benums}{\begin{enumerate*}}
\newcommand{\eenums}{\end{enumerate*}}
\newcommand{\benuma}{\begin{enumeratea}}
\newcommand{\eenuma}{\end{enumeratea}}
\newcommand{\benumar}{\begin{enumeratear}}
\newcommand{\eenumar}{\end{enumeratear}}

\newcommand{\PP}{\text{P}}
\newcommand{\R}{\mathbb{R}}
\newcommand{\N}{\mathbb{N}}
\newcommand{\C}{\mathbb{C}}
\newcommand{\E}{\mathbb{E}}
\newcommand{\HH}{\mathbb{H}}
\newcommand{\M}{\mathbb{M}}
\newcommand{\Rp}{\R_{\geq 0}}
\newcommand{\inv}{{-1}}
\newcommand{\normm}[1]{\lVert#1\rVert}
\newcommand{\normnull}{\left\lVert \cdot \right\rVert}
%\newcommand{\abs}[1]{\left\lvert #1\right\rvert}
\newcommand{\babs}[1]{\big\lvert #1\big\rvert}
%\newcommand{\prd}[1]{\left\langle#1\right\rangle}
\newcommand{\prd}[1]{\left\langle #1\right\rangle}
\newcommand{\bprd}[1]{\big\langle #1\big\rangle}
\newcommand{\Bprd}[1]{\Big\langle #1\Big\rangle}
\DeclarePairedDelimiter\prdd{\langle}{\rangle} 
\newcommand{\prdl}[1]{\prd{#1 ,\cdot\,}}
\newcommand{\prdr}[1]{\prd{\, \cdot, #1}}
\newcommand{\prdnull}{\left\langle \cdot \, , \cdot \right\rangle}
\newcommand{\Prd}[1]{\langle\!\langle#1\rangle\!\rangle}
\newcommand{\Prdnull}{\Prd{\cdot}}
\newcommand{\brck}[1]{\!\left[#1\right]}
\newcommand{\brckk}[1]{\!\left[\, #1\, \right]}
\newcommand{\prnt}[1]{\!\left(#1\right)}
\newcommand{\brce}[1]{\!\left\{ #1\right\}}
%\newcommand{\aqty}[1]{\left\langle #1\right\rangle}
%\newcommand{\brck}[1]{\left[#1\right]}
%\newcommand{\prnt}[1]{\left(#1\right)}
%\newcommand{\brce}[1]{\left\{ #1\right\}}
\newcommand{\varg}[1]{\!\left\langle#1\right\rangle\!}
\newcommand{\indt}{\hspace{7mm}}
\newcommand{\To}{\Rightarrow}
\newcommand{\longto}{\longrightarrow}
\newcommand{\Longto}{\Longrightarrow}
\newcommand{\Then}{\, \Rightarrow \,}
\newcommand{\Longthen}{ \ \Longrightarrow \ }
\newcommand{\notimplies}{\;\not\!\!\!\implies}

\newcommand{\vphi}{\varphi}
\renewcommand{\qedsymbol}{\rule{0.5em}{0.5em}}
\newcommand{\Z}{\mathbb{Z}}
\newcommand{\Q}{\mathbb{Q}}
\newcommand{\scrE}{\mathcal{E}}
\newcommand{\scrD}{\mathcal{D}}
\newcommand{\scrF}{\mathcal{F}}
\newcommand{\T}{\mathbb{T}}
\newcommand{\scr}[1]{\mathcal{#1}}
\newcommand{\scrr}[1]{\mathscr{#1}}
\newcommand{\fr}[1]{\mathfrak{#1}}
\newcommand{\nm}[1]{\textnormal{#1}}

\newcommand{\psigmm}{\psi_\gamma}
\newcommand{\Ro}{\R^3 - \qty{0}}
\newcommand{\dom}{\textnormal{dom}}
\newcommand{\psia}{\psi_\alpha}
\newcommand{\EE}{{\bf E}}
\newcommand{\test}[1]{C_c^\infty(#1)}
\newcommand{\loc}{\nm{loc}}
\newcommand{\odd}{\nm{odd}}
\newcommand{\even}{\nm{even}}
\newcommand{\vect}[1]{\textnormal{\textbf{#1}}}
\newcommand{\phii}{\varphi}
\newcommand{\one}{\textnormal{1}}
\newcommand{\onee}{\textnormal{1\,}}
\newcommand{\One}{{\bm{1}}}
\newcommand{\Onee}{{\bm{1}\,}}
\newcommand{\Vol}{\text{Vol}}

\newcommand{\eq}{\ = \ }
\newcommand{\eqh}{\, = \, }
\newcommand{\geql}{\ \geq \ }
\newcommand{\leql}{\ \leq \ }
\newcommand{\geqh}{\, \geq \, }
\newcommand{\leqh}{\, \leq \, }
%\newcommand{\equalq}{\quad = \quad}
%\newcommand{\qleq}{\quad \leq \quad}
%\newcommand{\qgeq}{\quad \geq \quad}
%\newcommand{\ssp}[1]{ \ {#1} \ }



\newcommand{\recip}[1]{\frac{1}{#1}}
\newcommand{\trecip}[1]{\tfrac{1}{#1}} 
\newcommand{\expp}[1]{\exp \prnt{#1}}
\newcommand{\qtext}{\quad \text}
%\newcommand{\qqtext}{\qquad \text}
\newcommand{\Qtext}{\quad\qtext}
\newcommand{\QQtext}[1]{\quad \qqtext{#1} \quad}
\newcommand{\given}{\, \vert \,}
\newcommand{\ggiven}{\, \big\vert \,}
\newcommand{\Given}{\, \Big\vert \,}
\newcommand{\GGiven}{\, \bigg\vert \,}
\newcommand{\longgiven}{\, \Bigg\vert \,}
\newcommand{\ninfty}{{-\infty}}
\newcommand{\zero}{{\, 0}}
\newcommand{\colorr}[1]{\textcolor{red}{#1}}
\newcommand{\coloro}[1]{\textcolor{orange}{#1}}
\newcommand{\colorb}[1]{\textcolor{blue}{#1}}
\newcommand{\colorrb}[1]{\textcolor{RoyalBlue}{#1}}

\newcommand{\colorm}[1]{\textcolor{magenta}{#1}}
\newcommand{\colormh}[1]{\textcolor{Mahogany}{#1}}
\newcommand{\colorfg}[1]{\textcolor{ForestGreen}{#1}}
\newcommand{\colorgy}[1]{\textcolor{gray}{#1}}
\newcommand{\colorg}[1]{\textcolor{green}{#1}}
\newcommand{\colort}[1]{\textcolor{teal}{#1}}
\newcommand{\colorp}[1]{\textcolor{purple}{#1}}
\newcommand{\colorc}[1]{\textcolor{cyan}{#1}}

\newcommand{\bb}[1]{\mathbb{#1}}
\newcommand{\Nzero}{\N \cup \brce{0}}
\newcommand{\ind}[2]{{#1=#2}^\infty}
\newcommand{\vech}{\accentset{\rightharpoonup}}
\newcommand{\sbullet}[1][.5]{\mathbin{\vcenter{\hbox{\scalebox{#1}{$\bullet$}}}}}
\newcommand{\sdot}{{\,\sbullet \,}}

\newcommand{\Span}{\textnormal{Span}}
\newcommand{\A}{\mathcal{A}}
\newcommand{\Sch}{\mathcal{S}}

\newcommand{\bd}{\partial}  
\newcommand{\eps}{\epsilon}
\newcommand{\veps}{\varepsilon}
\newcommand{\epsv}{\varepsilon}
\newcommand{\lam}{\lambda}
\newcommand{\Lam}{\Lambda}

\newcommand{\der}[1]{\frac{d}{d{#1}}}
\newcommand{\tder}[1]{\tfrac{d}{d{#1}}}
\newcommand{\coder}[1]{\frac{D}{d{#1}}}
\newcommand{\gcoder}[1]{\frac{\nab}{d{#1}}}
\newcommand{\Der}[2]{\frac{d{#1}}{d{#2}}}
\newcommand{\coDer}[2]{\frac{D{#1}}{d{#2}}}
\newcommand{\gcoDer}[2]{\frac{\nab{#1}}{d{#2}}}
\newcommand{\dder}[1]{\frac{d^{\,2}}{d{#1}^2}}
\newcommand{\tdder}[1]{\tfrac{d^{\,2}}{d{#1}^2}}
\newcommand{\DDer}[2]{\frac{d^{\,2}{#1}}{d{#2}^2}}

\newcommand{\nabder}[1]{\gcoder{#1}}
\newcommand{\nabDer}[2]{\gcoDer{#1}{#2}}

\newcommand{\pder}[1]{\frac{\partial}{\partial{#1}}}
\newcommand{\Pder}[2]{\frac{\partial{#1}}{\partial{#2}}}
\newcommand{\copder}[1]{\frac{D}{\partial{#1}}}
\newcommand{\coPder}[2]{\frac{D{#1}}{\partial{#2}}}

\newcommand{\ppder}[1]{\frac{\partial^2}{\partial{#1}^2}}
\newcommand{\PPder}[2]{\frac{\partial^2{#1}}{\partial{#2}^2}}


\newcommand{\floor}[1]{\lfloor{#1}\rfloor}
\newcommand{\ceil}[1]{\lceil{#1}\rceil}
\newcommand{\specT}{\sigma_{ap}(\textbf{T})}
\newcommand{\TT}{\textbf{T}}
\newcommand\restr[2]
  {{% we make the whole thing an ordinary symbol
  \left.\kern-\nulldelimiterspace % automatically resize the bar with \right
  #1 % the function
  \vphantom{\big|} % pretend it's a little taller at normal size
  \right|_{#2} % this is the delimiter
  }}


\newcommand{\specap}[1]{\sigma_{ap}{(#1)}}
\newcommand{\Nul}{\nm{Nul}\,}
\newcommand{\Ran}{\nm{Ran}\,}
\newcommand{\mtng}{martingale}
\newcommand{\cont}{continuous}
\newcommand{\diff}{differentiable}
\newcommand{\bdd}{bounded}
\newcommand{\dsp}{\displaystyle}

\newcommand\ShiftDown[2]{\raisebox{-#1}{\upshape\scriptsize #2}}


\newcommand\x{\times}
\newcommand\y{\cellcolor{green!10}}
\newcommand*\xbar[1]{%
  \hbox{%
    \vbox{%
      \hrule height 0.5pt % The actual bar
      \kern0.3ex%         % Distance between bar and symbol
      \hbox{%
        \kern-0.15em%      % Shortening on the left side
        \ensuremath{#1}%
        \kern-0.05em%      % Shortening on the right side
      }%
    }%
  }%
} 

\newcommand\reallywidehat[1]{\arraycolsep=0pt\relax%
 \begin{array}{c}
  \stretchto{
    \scaleto{
      \scalerel*[\widthof{\ensuremath{#1}}]{\kern-.5pt\bigwedge\kern-.5pt}
      {\rule[-\textheight/2]{1ex}{\textheight}} %WIDTH-LIMITED BIG WEDGE
    }{\textheight} % 
  }{0.5ex}\\           % THIS SQUEEZES THE WEDGE TO 0.5ex HEIGHT
  #1\\                 % THIS STACKS THE WEDGE ATOP THE ARGUMENT
  \rule{-1ex}{0ex}
  \end{array}
  }


\def\Xint#1{\mathchoice
{\XXint\displaystyle\textstyle{#1}}%
{\XXint\textstyle\scriptstyle{#1}}%
{\XXint\scriptstyle\scriptscriptstyle{#1}}%
{\XXint\scriptscriptstyle\scriptscriptstyle{#1}}%
\!\int}
\def\XXint#1#2#3{{\setbox0=\hbox{$#1{#2#3}{\int}$ }
\vcenter{\hbox{$#2#3$ }}\kern-.6\wd0}}
\def\ddint{\Xint=}
\def\dint{\Xint-}

\newcommand{\dt}[1]{%
  \accentset{\mbox{\large\bfseries .}}{#1}}
\newcommand{\ddt}[1]{%
  \accentset{\mbox{\large\bfseries .\hspace{-0.25ex}.}}{#1}}
  

\DeclarePairedDelimiter\paren{(}{)} 


\DeclareFontFamily{U}{mathx}{}
\DeclareFontShape{U}{mathx}{m}{n}{ <-> mathx10 }{}
\DeclareSymbolFont{mathx}{U}{mathx}{m}{n}
\DeclareFontSubstitution{U}{mathx}{m}{n}

\DeclareMathSymbol{\widechecksym}{\mathord}{mathx}{"71}
\newcommand\lowerwidechecksym{%
  \text{\smash{\hspace*{-0.25ex}\raisebox{-1.1ex}{%
    $\widechecksym$}}}}
\newcommand\ch[1]{%
  \mathchoice
    {\accentset{\displaystyle\lowerwidechecksym}{#1}}
    {\accentset{\textstyle\lowerwidechecksym}{#1}}
    {\accentset{\scriptstyle\lowerwidechecksym}{#1}}
    {\accentset{\scriptscriptstyle\lowerwidechecksym}{#1}}
}

%\DeclareMathAccent{\what}{0}{mathx}{"70}
\DeclareMathAccent{\widecheck}{0}{mathx}{"71}
\newcommand{\wc}{\widecheck}


%\DeclareMathSymbol{\widehatsym}{\mathord}{mathx}{"70}
%\newcommand\lowerwidehatsym{%
%  \text{\smash{\hspace*{-0.25ex}\raisebox{-1.1ex}{%
%    $\widehatsym$}}}}
%\newcommand\wh[1]{%
%  \mathchoice
%    {\accentset{\displaystyle\lowerwidehatsym}{#1}}
%    {\accentset{\textstyle\lowerwidehatsym}{#1}}
%    {\accentset{\scriptstyle\lowerwidehatsym}{#1}}
%    {\accentset{\scriptscriptstyle\lowerwidehatsym}{#1}}
%}
   
   
\DeclareMathOperator*{\LLlim}{\emph{L}^2-lim}% Define \Lim

%----------------------------------------------------------------------------------------------------------------
%	E N V I R O N M E N T S
%----------------------------------------------------------------------------------------------------------------

\newenvironment{ppr}[2]
{ {\bf #1\\}{- #2} \vspace{.3cm}  \\
}

%-------------------------------------------------------------------------------------------------------------  

\newenvironment{itemize*}%
  { \vspace{0mm}
  \begin{itemize}%
    \setlength{\itemsep}{0em}%
    \setlength{\parskip}{1em}}%
  {\end{itemize}}
  
%-------------------------------------------------------------------------------------------------------------  

\newenvironment{enumerate*}%
  {\begin{enumerate}[label=\nm{\roman*)}]%
  }
  {\end{enumerate}
  }
  
%-------------------------------------------------------------------------------------------------------------  

\newenvironment{enumeratea}%
  {\begin{enumerate}[label=\nm{\alph*)}]%
  }
  {\end{enumerate}
  }
  

%-------------------------------------------------------------------------------------------------------------  

\newenvironment{enumeratear}%
  {\begin{enumerate}[label=\nm{\arabic*)}]%
  }
  {\end{enumerate}
  }
  
  
%-------------------------------------------------------------------------------------------------------------
  
  \newmdenv[
  topline=false,
  bottomline=false,
  rightline=false,
  skipabove=\topsep,
  skipbelow=\topsep,
  leftmargin=0pt,
  %rightmargin=0pt,
  innertopmargin=10pt,
  %innerbottommargin=-10pt
  ]{siderules}
  
%-------------------------------------------------------------------------------------------------------------  
  
  \newenvironment{siderules*}
    {\begin{siderules} \vspace{0cm}
    }
    {\end{siderules}
    }
    
  
%-------------------------------------------------------------------------------------------------------------
  
  \newenvironment{workout}
    {\begin{adjustwidth}{1cm}{}
     %\vspace{10pt}
      \begin{siderules}
      }
      {\end{siderules}
    \end{adjustwidth}
    }   
 
%-------------------------------------------------------------------------------------------------------------
  
  \newenvironment{workout*}
    {\vspace{10pt}
    \begin{adjustwidth}{1cm}{}
      }
      {
    \end{adjustwidth}
    \vspace{10pt}
    }

%-------------------------------------------------------------------------------------------------------------

\newenvironment{chapabstract}{%
    \begin{center}%
      \bfseries Chapter Abstract
    \end{center}}%
   {\par}
   
%-------------------------------------------------------------------------------------------------------------
   
   
\usepackage{enumitem}

\newlist{mylista}{enumerate*}{1}
\setlist[mylista]{label=(\arabic*),itemjoin=\quad }

\newlist{mylistr}{enumerate*}{1}
\setlist[mylistr]{label=\nm{(\roman*)},itemjoin=\quad}

%-------------------------------------------------------------------------------------------------------------

\newenvironment{myfont}[2][]{\csname#2\endcsname[#1]}{}
\newcommand{\Alpine}[1][]{\fontfamily{Alpine}#1\selectfont}
\newcommand{\Helvet}[1][]{\fontfamily{ppl}#1\selectfont}



%----------------------------------------------------------------------------------------------------------------
%	T H E O R E M    S T Y L E
%----------------------------------------------------------------------------------------------------------------

\newtheoremstyle{break}
{1.5em}
{1.5em}
{\itshape}
{}
{\bfseries}
{.}
{\newline}
{}
\theoremstyle{break}

\newtheoremstyle{newstyle}      
{1.5em} %Aboveskip 
{1.5em} %Below skip
{\itshape} %Body font e.g.\mdseries,\bfseries,\scshape,\itshape
{} %Indent
{\bfseries} %Head font e.g.\bfseries,\scshape,\itshape
{.} %Punctuation afer theorem header
{1em} %Space after theorem header
{\thmname{#1}\thmnumber{ #2}\thmnote{ (#3)}} %Heading

\newtheoremstyle{newstyle2}      
{1.5em} %Aboveskip 
{1.5em} %Below skip
{\mdseries} %Body font e.g.\mdseries,\bfseries,\scshape,\itshape
{} %Indent
{\bfseries} %Head font e.g.\bfseries,\scshape,\itshape
{.} %Punctuation afer theorem header
{.3cm} %Space after theorem header
{\thmname{#1}\thmnumber{ #2}\thmnote{ (#3)}} %Heading

\newtheoremstyle{newstyle3}      
{1.5em} %Aboveskip 
{1.5em} %Below skip
{\mdseries} %Body font e.g.\mdseries,\bfseries,\scshape,\itshape
{} %Indent
{\bfseries} %Head font e.g.\bfseries,\scshape,\itshape
{.} %Punctuation afer theorem header
{\newline} %Space after theorem header
{\thmname{#1}\thmnumber{ #2}\thmnote{ (#3)}} %Heading

\newtheoremstyle{newstyle4}      
{1.5em} %Aboveskip 
{1.5em} %Below skip
{\itshape} %Body font e.g.\mdseries,\bfseries,\scshape,\itshape
{} %Indent
{\bfseries} %Head font e.g.\bfseries,\scshape,\itshape
{.} %Punctuation afer theorem header
{.3cm} %Space after theorem header
{\thmname{#1}\thmnumber{ #2}\thmnote{ (#3)}} %Heading

\newtheoremstyle{newstyle5}      
{1em} %Aboveskip 
{1em} %Below skip
{\mdseries} %Body font e.g.\mdseries,\bfseries,\scshape,\itshape
{} %Indent
{\bfseries} %Head font e.g.\bfseries,\scshape,\itshape
{\\ } %Punctuation afer theorem header
{.3cm} %Space after theorem header
{\thmname{#1}\thmnumber{ #2}\thmnote{ (#3)}} %Heading


%\theoremstyle{newstyle}
%\newtheorem{thm}{Theorem}[Section]
%\theoremstyle{newstyle}
%\newtheorem{cor}{Corollary}[thm]
%\newtheorem{lmm}[thm]{Lemma}
%\newtheorem{defn}[thm]{Definition}
%\newtheorem{rmk}[thm]{Remark}
%\newtheorem{pps}[thm]{Proposition}
%\theoremstyle{definition}
%\newtheorem{exm}[thm]{Example}









%----------------------------------------------------------------------------------------------------------------
%	C O U N T E R
%----------------------------------------------------------------------------------------------------------------


%--------Theorem Counter----------------------------------------------------------------
\theoremstyle{newstyle}
\newtheorem{thm}{Theorem}[section]   		%numbers after .
%\newtheorem{thm}{Theorem}[chapter]

%-----------------------------------------------------------------------------------------------

\setcounter{section}{0}
\setcounter{tocdepth}{2}
\setcounter{secnumdepth}{5}


\counterwithin{thm}{section} 					%numbers before .
%\counterwithin{thm}{chapter}

\counterwithin{equation}{section}
%\counterwithin{equation}{chapter}
%\counterwithin*{equation}{section}








%----------------------------------------------------------------------------------------------------------------
%	N E W    T H E O R E M
%----------------------------------------------------------------------------------------------------------------

\theoremstyle{newstyle}
\newtheorem*{thm*}{Theorem}

\theoremstyle{newstyle4}
\newtheorem{thmx}{Theorem}

\theoremstyle{newstyle}
\newtheorem{cor}[thm]{Corollary}

\theoremstyle{newstyle4}
\newtheorem{corx}{Corollary}

\theoremstyle{newstyle}
\newtheorem{lmm*}{Lemma}

\theoremstyle{newstyle}
\newtheorem{lmm}[thm]{Lemma}

\theoremstyle{newstyle2}
\newtheorem{defn}[thm]{Definition}

\theoremstyle{newstyle3}
\newtheorem{defn*}{Definition}

\theoremstyle{newstyle2}
\newtheorem{rmk}[thm]{Remark}

\theoremstyle{newstyle2}
\newtheorem*{rmk*}{Remark}

\theoremstyle{newstyle2}
\newtheorem{note}[thm]{Note}

\theoremstyle{newstyle}
\newtheorem{pps}[thm]{Proposition}

\theoremstyle{newstyle}
\newtheorem*{pps*}{Proposition}

\theoremstyle{newstyle4}
\newtheorem{ppsx}{Proposition}

\theoremstyle{newstyle2}
\newtheorem{exm}[thm]{Example}

\theoremstyle{newstyle3}
\newtheorem{exm*}{Example}

\theoremstyle{newstyle2}
\newtheorem{smm}{Summary}

\theoremstyle{newstyle2}
\newtheorem{nota}[thm]{Notation}




%\theoremstyle{newstyle}
%\newtheorem*{thm}{Theorem}

%\newtheorem*{qst}{\textcolor{red} {Question}}
%\newtheorem*{qstn}{\textcolor{red} {Question}}

\newtheorem*{qst}{{Question}}
\newtheorem*{qstn}{{Question}}
\theoremstyle{newstyle2.5}
\newtheorem*{ans}{\textcolor{red} {Answer}}


\theoremstyle{newstyle2}
\newtheorem{pblm}[thm]{Problem}
%\newtheorem{pblm}{Problem}[chapter]

\theoremstyle{newstyle2}
\newtheorem{exc}[thm]{Exercise}
%\newtheorem{pblm}{Problem}[chapter]

\theoremstyle{newstyle2}
\newtheorem*{sol}{Solution}

\theoremstyle{newstyle2}
\newtheorem*{answ}{Answer}


% Define newstyle
\newtheoremstyle{named}
{1em}{}{\itshape}{}{\bfseries}{.}{0.5em}{\thmnote{#3}#1} 


%\theoremstyle{definition}
\theoremstyle{named} 
% Set up the header and footer
\newtheorem*{namedthm}{}



% Define newstyle
\newtheoremstyle{named2}
{1em}{}{\itshape}{}{\bfseries}{}{0.5em}{\thmnote{#3}#1} 

%\theoremstyle{definition}
\theoremstyle{named2} 
\newtheorem*{namedthmno}{}



% Define newstyle
\newtheoremstyle{named3}
{1em}{}{\mdseries}{}{\bfseries}{}{1em}{\thmnote{#3}#1} 

%\theoremstyle{definition}
\theoremstyle{named3} 
\newtheorem*{namedexm}{}

%\newmdtheoremenv{theo}{Theorem}







%----------------------------------------------------------------------------------------------------------------
%	P A G E   H E A D E R
%----------------------------------------------------------------------------------------------------------------

\pagestyle{fancy}
\fancyhf{}
\cfoot{\thepage} 
\fancyhead[L]{{\footnotesize {\it \rightmark}}}
%\fancyhead[R]{\thepage}
\renewcommand{\headrulewidth}{0pt}

\setlength\parindent{7mm} % Removes all indentation from paragraphs



%--------------------chapter format---------------------

% CENTER WITHOUT NUMBER
%\titleformat{\chapter}[display]
%{\centering \normalfont\bfseries}{}{0pt}{\Huge}

% CENTER WITH NUMBER
\titleformat{\chapter}[hang]
{\centering \Huge\bfseries}{\thechapter{. }}{0pt}{\Huge\bfseries}

% LEFT WITH NUMBER
%\titleformat{\chapter}[hang]
%{\Huge\bfseries}{\thechapter{. }}{0pt}{\Huge\bfseries}

% RIGHT WITH NUMBER
%\titleformat{\chapter}[hang]
%{\filleft \Huge\bfseries}{\thechapter{. }}{0pt}{\Huge\bfseries}


%  for Papers Summary
%  CENTER WITHOUT NUMBER, NORMAL FONT
%\titleformat{\chapter}[display]
%{\centering \normalfont\mdseries}{}{0pt}{\LARGE}




%---------------------section format---------------------

%\renewcommand{\thesection}{\arabic{section}.\arabic{subsection}}
\renewcommand{\thesection}{\arabic{section}}
% Nothing
  


%-----------------------------------------------------------------------------------------------------------------  
%\pagestyle{fancy}
%\fancyhf{}
%\cfoot{} 
%\fancyhead[L]{\rightmark}
%\fancyhead[R]{\thepage}
%\renewcommand{\headrulewidth}{0.5pt}
%
%\setlength\parindent{15pt} % Removes all indentation from paragraphs


%\counterwithin{equation}{section}
%\newcounter{example}
%\counterwithin{example}{section}
%\newenvironment{example}[1][]{%
%\stepcounter{example}%
%\par\vspace{5pt}\noindent
%\fbox{\textbf{Example~\thesection.\theexample}}%
%\hrulefill\par\vspace{10pt}\noindent\rmfamily}%
%{\par\noindent\hrulefill\vrule width10pt height2pt depth2pt\par}



%----------------------------------------------------------------------------------------------------------------
%	D O C U M E N T
%----------------------------------------------------------------------------------------------------------------


\usepackage{graphicx}


\def\rcurs{{\mbox{$\resizebox{.16in}{.08in}{\includegraphics{ZScriptR}}$}}}
\def\brcurs{{\mbox{$\resizebox{.16in}{.08in}{\includegraphics{ZBoldR}}$}}}
\def\hrcurs{{\mbox{$\hat \brcurs$}}}

\begin{document}

\def\dnicefrac#1#2{ 
	\raise.5ex\hbox{$#1$} 
	\kern-.1em/\kern-.15em 
	\lower.25ex\hbox{$#2$}}
\title{RESEARCH SUPPLEMENTARY NOTES}

\begin{onehalfspacing}

\def\dnicefrac#1#2{ 
	\raise.5ex\hbox{$#1$} 
	\kern-.1em/\kern-.15em  t
	\lower.25ex\hbox{$#2$}}
	
\newcommand*\widefbox[1]{\fbox{\hspace{2em}#1\hspace{2em}}}
	

%------------------------------------------------------------------------------ T I T L E

%\centerline{\Huge \bf {Research Side Notes}} \vspace{.5cm}
%\maketitle
\tableofcontents

\newpage

%------------------------------------------------------------------------------ B O D Y


%
%\section{First equation}
%Some introductory text...
%\begin{example}
%\begin{equation}
%    f(x)=\frac{x}{1+x^2}
%\end{equation}
%\end{example}
%
%\subsection{More detail}
%\begin{example}
%Here we discuss
%\begin{equation}
%    f(x)=\frac{x+1}{x-1}
%\end{equation}
%... but don't say anything
%\end{example}
%
%\subsubsection{Even more detail}
%\begin{example}
%\begin{equation}
%    f(x)=\frac{x^2}{1+x^3}
%\end{equation}
%\begin{equation}
%    f(x+\delta x)=\frac{(x+\delta x)^2}{1+(x+\delta x)^3}
%\end{equation}
%\end{example}
%
%\section{Third equation}
%\begin{example}
%The following function...
%\begin{equation}
%    f_1(x)=\frac{x+1}{x-1}
%\end{equation}
%..is a function
%\begin{equation}
%    f_2(x)=\frac{x^2}{1+x^3}
%\end{equation}
%\begin{equation}
%    f_3(x)=\frac{3+ x^2}{1-x^3}
%\end{equation}
%\end{example}

%  \include{Notebook}
%  \include{Ch_ProbBasics}
%  \include{Ch_DirichletForm}
%  \include{Ch_SDE}
%  \include{Ch_Measure}
%  \include{Ch_Topology}
%  \include{Ch_DiffGeo}
%  \include{Ch_MathPhysics}
%  \include{Ch_PDE}
%  \import{Old Files/}{Ch_ProbBasics}
%  \import{New Files/}{DirichletForm}
  
  
%------------------------TOPOLOGY------------------------

%  \import{Topology/}{Topology}
%  \import{Topology/Wilansky/}{Semimetric}
%  \import{Topology/Wilansky/}{BaseSubbase}
%  \import{Topology/Wilansky/}{Sequence}
%  \import{Topology/Wilansky/}{Filter}
%  \import{Topology/Wilansky/}{Separable}
%  \import{Topology/}{Compact}
%  \import{Topology/}{StoneWeier}
%  \import{Topology/Wilansky/}{WeakTopology}
%  \import{Topology/Wilansky/}{ProductTopology}
  

%------------------------REAL------------------------

%  \import{Real/}{Chapter}  
%  \import{Real/}{HahnBanach}    
%  \import{Real/}{Baire}  
%  \import{Real/}{BairetoBanach} 
%  \import{Real/}{ContSpace} 
%  \import{Real/}{Embedding}
%  \import{Real/}{TopoVector}
%  \import{Real/}{LCH}
%  \import{Real/}{SobolevSpace}
%  \import{Real/}{TVS} 
%  \import{Real/}{Mollifier}


%  \import{Real/Strichartz}{Ch1} 
%  \import{Real/Strichartz}{Ch2}
%  \import{Real/Strichartz}{Ch3}   
%  \import{Real/Strichartz}{Ch4}
%  \import{Real/Strichartz}{Ch5}
%  \import{Real/Strichartz}{Ch6}
%  \import{Real/Strichartz}{Ch7}
%  \import{Real/Strichartz}{Ch8}
          

%------------------------MEASURE------------------------

%  \import{Measure/}{LebesgueDiff}
%  \import{Measure/}{Convergence}
 
%  \import{Measure/}{ConvInDist}
%  \import{Measure/}{SpaceMeasure}
%  
%  \import{Measure/}{MeasureOnTopo}
%  
%  \import{Measure/}{MeasWeakConv}
%  \import{Measure/}{WeakCompact}
%  
%  \import{Measure/Radon}{Radon}
%  \import{Measure/Radon/}{Dual}
%  \import{Measure/Radon/}{Positive}
%  \import{Measure/Radon/}{Regularity}


%------------------------FUNCTIONAL------------------------

%  \import{Functional}{UnboundedOperator} 
%  \import{Functional/}{HarmonicOscillator} 
%  \import{Functional/ReedSimonII/IX/}{Ch IX} 
  \import{Functional/ReedSimonII/X/}{Ch X} 
  \import{Functional/Albeverio/I/}{Ch1} 
%  \import{Functional/AkhiezerGlazman/Ch8/}{Ch8} 
%  \import{Functional/}{KreinFormula} 

%------------------------QUANTUM------------------------

%  \import{Physics/Quantum/Hall}{Ch2}
%  \import{Physics/Quantum/Hall/}{Ch3}

%------------------------PROB------------------------

%  \import{Prob/}{CLT}  
%  \import{Prob/}{Conditional1}
%  \import{Prob/}{Kolmogorov}
%  \import{Prob/BM}{BM}
%  \import{Prob/}{MarkovDiscrete}
%  \import{Prob/}{MarkovCont}
%  \import{Prob/}{MarkovProperty}
%  \import{Prob/}{MarkovProcess}
%  \import{Prob/Basics}{RV}
%  \import{Prob/Basics}{SumRV}
%  \import{Prob/Basics}{GaussianVector}
%  \import{Prob/Gaussian}{GaussianVector}
%  \import{Prob/Gaussian}{GaussianMeasure}
%  \import{Prob/BM}{BMnGaussian}

%  \import{Prob/Martingale}{Krickeberg}
%  \import{Prob/Martingale}{Martingale}
%  \import{Prob/Martingale}{MartingaleL2}
%  \import{Prob/Martingale}{ConvThm}
%  \import{Prob/Martingale}{UI}
%  \import{Prob/Martingale}{UImartingale}
%  \import{Prob/Martingale}{Summary}
%  \import{Measure/}{RadonNikodym}
%  \import{Prob/Martingale}{UI}

%  \import{Prob/MarkovChains/Norris}{Ch1}
%  \import{Prob/MarkovChains/Norris}{Ch2}
%  \import{Prob/MarkovChains/Norris}{Ch3}    
%    
%  \import{Prob/MarkovChains}{Properties}
%  \import{Prob/MarkovChains}{Transience}
%  \import{Prob/MarkovChains}{Recurrence}
%  \import{Prob/MarkovChains}{Potential}
%  \import{Prob/MarkovChains}{TranPotential}

%  \import{Prob/LoopSoup}{Ch1} 
%  \import{Prob/LoopSoup}{Ch2} 
%  \import{Prob/LoopSoup}{Ch3} 
%  \import{Prob/LoopSoup}{Ch4} 

%  \import{Prob/LargeDeviations}{Ch1}
%  \import{Prob/LargeDeviations}{Ch2}
%  \import{Prob/LargeDeviations}{Ch3}


%  \import{Prob/Feller}{Ch1}  
%  \import{Prob/Feller}{Ch2}   
%  \import{Prob/Feller}{Ch3}  

%  \import{Prob/BM/MorPer}{Ch1}
%  \import{Prob/BM/MorPer}{Ch2}
%  \import{Prob/BM/MorPer}{Ch3}
%  \import{Prob/BM/MorPer}{Ch4}
%  \import{Prob/BM/MorPer}{Ch5}
%  \import{Prob/BM/MorPer}{Ch6}
%  \import{Prob/BM/MorPer}{Ch7}
%  \import{Prob/BM/MorPer}{Ch8}  
%  \import{Prob/BM/MorPer}{Ch9}  
%  \import{Prob/BM/MorPer}{Ch10}  
%  \import{Prob/BM/MorPer}{Ch11}  
  
    
%   \import{Prob/}{MarkovDiscrete}    
%
%   \import{Prob/Gaussian/Eldredge}{Ch1}    
%   \import{Prob/Gaussian/Eldredge}{Ch2}    
%   \import{Prob/Gaussian/Eldredge}{Ch3}
%   \import{Prob/Gaussian/Eldredge}{Ch4}
%   \import{Prob/Gaussian/Eldredge}{Ch5}
%   \import{Prob/Gaussian/Eldredge}{Ch6}    
%   \import{Prob/Gaussian/Eldredge}{Ch7}  
%   \import{Prob/Gaussian/Eldredge}{Ch8}  
%   \import{Prob/Gaussian/Eldredge}{Summary}     


%   \import{Prob/Wiener/Driver}{Ch2}      
%   \import{Prob/Wiener/Driver}{Ch3}  
%   \import{Prob/Wiener/Driver}{Ch4}  
%   \import{Prob/Wiener/Driver}{Ch5}
%   \import{Prob/Wiener/Driver}{Ch6}
%   \import{Prob/Wiener/Driver}{Ch7}
%   \import{Prob/Wiener/Driver}{Ch8}


              
%   \import{Prob/Stochastic/Emery}{Ch3}     

%   \import{Prob/MarkovOperator/}{Sanchez}

%   \import{Prob/RogersWilliams/III}{MarkovProcesses}
%   \import{Prob/RogersWilliams/IV}{Ito}
%   \import{Prob/RogersWilliams/V}{[SDE}
%   \import{Prob/RogersWilliams/VI}{Dual}
%   \import{Prob/RogersWilliams/VI}{Excursions}


%   \import{Prob/Lawler}{Ch9}

%   \import{Prob/BM/Chung/I}{Ch1}
%   \import{Prob/BM/Chung/I}{Ch2}
%   \import{Prob/BM/Chung/I}{Ch5}
%   \import{Prob/BM/Chung/I}{Ch6}
%   \import{Prob/BM/Chung/I}{Ch7}
%   \import{Prob/BM/Chung/I}{Ch8}
%   \import{Prob/BM/Chung/I}{Ch10}
%   \import{Prob/BM/Chung/I}{Ch11}   
%   \import{Prob/BM/Chung/I}{Ch12}   

%   \import{Prob/BM/Chung/II}{Ch1}   
%   \import{Prob/BM/Chung/II}{Ch2}
%   \import{Prob/BM/Chung/II}{Ch3}
%   \import{Prob/BM/Chung/II}{Ch4}
%   \import{Prob/BM/Chung/II}{Ch5}                 
%   \import{Prob/BM/Chung/II}{Ch6}   

%   \import{Prob/BM/Chung/III}{Ch1}  
   
%   \import{Prob/Oksendal}{Ch1}      
%   \import{Prob/Oksendal}{Ch3}      
%   \import{Prob/Oksendal}{Ch5}                  

%   \import{Prob/OksendalHolden}{Ch1}                 

%   \import{Prob/Excursion/Blumenthal}{Ch II}         
%   \import{Prob/Excursion/Blumenthal}{Ch III}            
%   \import{Prob/Excursion/Blumenthal}{Ch IV}               
%   \import{Prob/Excursion/Blumenthal}{Ch V}
%   \import{Prob/Excursion/Blumenthal}{Ch VI}
%   \import{Prob/Excursion/Blumenthal}{Ch VII}

%   \import{Prob/Excursion/RevuzYor}{Ch I}
%   \import{Prob/Excursion/RevuzYor}{Ch III}
%   \import{Prob/Excursion/RevuzYor}{Ch IV}         
%   \import{Prob/Excursion/RevuzYor}{Ch VI}
%   \import{Prob/Excursion/RevuzYor}{Ch XI}
%   \import{Prob/Excursion/RevuzYor}{Ch XII}

%   \import{Prob/Liggett}{Ch3}    

%   \import{Prob/ChungWalsh/}{Ch3}    

%   \import{Prob/KarlinTaylor/Ch15/}{Diffusion} 

%------------------------DIFFUSION------------------------

%  \import{Diffusion/Bass}{Part1}
%  \import{Diffusion/Bass}{Part2}
%  \import{Diffusion/Bass}{Part3}
%  \import{Diffusion/Bass}{Part4}
%  \import{Diffusion/Bass}{Part5}
%  \import{Diffusion/Bass}{Part6}
%  \import{Diffusion/Bass}{Part7}
%  \import{Diffusion/Bass}{Part8}
  
%  \import{Diffusion/Baudoin}{Ch1}
%  \import{Diffusion/Baudoin}{Ch2}
%  \import{Diffusion/Baudoin}{Ch3}
%  \import{Diffusion/Baudoin}{Ch4}
%  \import{Diffusion/Baudoin}{Ch6}
%  \import{Diffusion/Baudoin}{Summary}

%  \import{Diffusion/SDE}{Chapter}  
%  \import{Diffusion/SDE}{Construction}
%  \import{Diffusion/SDE}{Girsanov}
%  \import{Diffusion/SDE}{Ito}
%  \import{Diffusion/SDE}{MAF}
%  \import{Diffusion/SDE}{Quadratic}



%------------------------COMPLEX------------------------

%  \import{Complex/}{Elementary}
%  \import{Complex/}{Integral}
%  \import{Complex/}{Singularities}
%  \import{Complex/}{MaxMod}
%  \import{Complex/}{Compact}
%  \import{Complex/}{Runge}
%  \import{Complex/}{Harmonic}
%  \import{Complex/}{Entire}



%------------------------DIFF GEO------------------------

%  \import{DiffGeo/Weintraub}{Ch1}
%  \import{DiffGeo/Weintraub}{Ch2}
%  \import{DiffGeo/Weintraub}{Ch3}
%  \import{DiffGeo/Weintraub}{Ch4}
%  \import{DiffGeo/Weintraub}{Ch5}
%  \import{DiffGeo/Weintraub}{Ch6}


%  \import{DiffGeo/doCarmoEuclidean}{Ch1}
%  \import{DiffGeo/doCarmoEuclidean}{Ch2}
%  \import{DiffGeo/doCarmoEuclidean}{Ch3} 
%  \import{DiffGeo/doCarmoEuclidean}{Ch4}
%  \import{DiffGeo/doCarmoEuclidean}{Ch5}
%
%  \import{DiffGeo/doCarmo}{Supp}
%  \import{DiffGeo/doCarmo}{Ch0}
%  \import{DiffGeo/doCarmo}{Ch1}
%  \import{DiffGeo/doCarmo}{Ch2}
%  \import{DiffGeo/doCarmo}{Ch3}
%  \import{DiffGeo/doCarmo}{Ch4}
%  \import{DiffGeo/doCarmo}{Ch5}
%  \import{DiffGeo/doCarmo}{Ch6}
%  \import{DiffGeo/doCarmo}{Summary}


%  \import{DiffGeo/Spivak/Vol1/}{Ch3}
%  \import{DiffGeo/Spivak/Vol1/}{Ch4} 				%tensor
%  \import{DiffGeo/Spivak/Vol1/}{Ch5}
%  \import{DiffGeo/Spivak/Vol1/}{Ch6}
%  \import{DiffGeo/Spivak/Vol1/}{Ch7}				%diff form
%  \import{DiffGeo/Spivak/Vol1/}{Ch8}
%  \import{DiffGeo/Spivak/Vol1/}{Summary}
%  
%  \import{DiffGeo/Spivak/Vol2}{Ch3-4}
%  \import{DiffGeo/Spivak/Vol2}{Ch5}
%  \import{DiffGeo/Spivak/Vol2}{Ch6}

%  \import{DiffGeo/TensorCal/}{Title}
%  \import{DiffGeo/TensorCal/}{Video20}
%  \import{DiffGeo/TensorCal/}{Video22}

  \import{DiffGeo/Lee}{Lee}

%  \import{DiffGeo/}{Hodge}



%------------------------DIRICHLET------------------------

%  \import{DirichletForms/Fukushima}{Ch1}    
%  \import{DirichletForms/Fukushima}{Ch2}    
%  \import{DirichletForms/Fukushima}{Ch5}    
  
%  \import{DirichletForms/BouleauHirsch/}{I}    
  
%  \import{DirichletForms/MaRockner/}{I}    
  
%  \import{DirichletForms/}{Dirichlet}
%  \import{PaperSummary/Papers}{[Getoor] Transience}

%  \import{DirichletForms/}{Friedrichs}
  



%------------------------EM------------------------

%  \import{EM/}{Ch1}
%  \import{EM/}{Ch2}


%------------------------ODE------------------------

%  \import{ODE/}{Ch1}
%  \import{ODE/}{Ch2}
%  \import{ODE/}{Ch3}
%  \import{ODE/}{Ch5}
  
  

%------------------------PDE------------------------

 \import{PDE/}{GreenFunction}
%% \import{PDE/}{Harmonic}
% \import{PDE/HanLin}{Ch1}

% \import{PDE/Evans/Ch2/}{Ch2}

% \import{PDE/Strauss}{Ch1}
% \import{PDE/Strauss}{Ch2}
% \import{PDE/Strauss}{Ch3}
% \import{PDE/Strauss}{Ch4}
% \import{PDE/Strauss}{Ch7}
% \import{PDE/Strauss}{Ch12}
  

%------------------------LIE GROUP------------------------

%  \import{LieGroup/}{Ch1}
%  \import{LieGroup/}{Ch2}
%  \import{LieGroup/}{Ch3}
%  \import{LieGroup/}{Ch4}
%  \import{LieGroup/}{Ch5}
%  \import{LieGroup/}{Ch6}
%  \import{LieGroup/}{Ch7}
%  \import{LieGroup/}{Ch8}
%  \import{LieGroup/}{Ch9}
%  \import{LieGroup/}{Summary}      



%------------------------LINEAR ALG------------------------

%  \import{LinearAlg/}{Affine}       

%  \import{Etc/}{Chapter}
%  \import{ZOld Files/}{Ch_Diffusion}



%------------------------LECTURE NOTES------------------------

%  \import{LectureNotes/}{280B}
%  \import{LectureNotes/}{280C}
%  \import{LectureNotes/}{286}
%  \import{LectureNotes/}{294}


%------------------------SPECIAL------------------------ 

%  \import{SpecialTopics}{Percolation}
%  \import{SpecialTopics}{Ising}

%  \import{SpecialTopics/Mechanics/}{Ch1}
%  \import{SpecialTopics/Mechanics/}{Ch2}
%  \import{SpecialTopics/Mechanics/}{Appendix}


%------------------------NUMERICAL------------------------
		
%  \import{Numerical/Hildebrand/}{Ch7}
%  \import{Numerical/Hildebrand/}{Ch8}
%  \import{Numerical/}{GaussianQuadrature}



%------------------------STATISTICS------------------------

%  \import{Statistics/}{SamplingEstimator}



%------------------------RANDOM------------------------

%  \chapter*{Random Notes}
\vspace{-1.5cm}
\[
  \finline[\fontsize{11}{10}]{Helvet}
  {\emph{- A space for random ideas that cross my mind -}}
\] \vspace{-0.5cm}


%================================================================


%  \import{ZRandom/}{Hessian}
%
%  \import{ZRandom/}{Covariance}
%
%  \import{ZRandom/}{RW}
%  
%  \import{ZRandom/}{SumRV}
%
%  \import{ZRandom/}{Bivariate}
%
%  \import{ZRandom/}{SumRV2}
%
%  \import{ZRandom/}{Quadratic}
%
%  \import{ZRandom/}{BoundedMetric}
%  
%  \import{ZRandom/}{Complex}
%
%  \import{ZRandom/}{TeachReal}
%  
%  \import{ZRandom/}{Norm}
%
%  \import{ZRandom/}{MeanMed}
%
%  \import{ZRandom/}{SDMean}
%
%  \import{ZRandom/}{CondExp}
%%  
%  \import{ZRandom/}{Grad}
%
%  \import{ZRandom/}{LimitMulti}
%  
%  \import{ZRandom/}{ProductSpace}  
%
%  \import{ZRandom/}{WeakTopo}  
%  
%  \import{ZRandom/}{ConvDis}
%
%  \import{ZRandom/}{SpaceCont}
%  
%  \import{ZRandom/}{C0Cc}
%
%  \import{ZRandom/}{SpaceProb}
%
%  \import{ZRandom/}{Hessian2}
%  
%  \import{ZRandom/}{Compact}
%
%  \import{ZRandom/}{Diff}
%
%  \import{ZRandom/}{ComplexDiff}
%
%  \import{ZRandom/}{Urysohn}
%  
%  \import{ZRandom/}{SII}
%  
%  \import{ZRandom/}{Arclength}
%
%  \import{ZRandom/}{Conservative}
%  
%  \import{ZRandom/}{VectorCalThm}
%  
%  \import{ZRandom/}{GradCurlDiv}
%  
%  \import{ZRandom/}{GreenGaussStokes}
%
%  \import{ZRandom/}{DiffForm}
%  
%  \import{ZRandom/}{ProductVector}
%
%  \import{ZRandom/}{Parametrize}
%  
%  \import{ZRandom/}{ExactSeq}
%
%  \import{ZRandom/}{Hamil}
%
%  \import{ZRandom/}{Implicit}
%
%  \import{ZRandom/}{TangentVec}
%
%  \import{ZRandom/}{Curves}
%  
%  \import{ZRandom/}{Quaternions}
%
%  \import{ZRandom/}{InnerProduct}
  
%  \import{ZRandom/}{LargeDev}

%  \import{ZRandom/}{ExistUnique}
  
%  \import{ZRandom/}{CrossProduct}

%  \import{ZRandom/}{BrownianDiff}
  
%  \import{ZRandom/}{Martingale}
  
%  \import{ZRandom/}{MonotoneClass}
  
%  \import{ZRandom/}{PointSeparate}

%  \import{ZRandom/}{Diff2}

%  \import{ZRandom/}{DetEq}
  
%  \import{ZRandom/}{GraphTransf}

%  \import{ZRandom/}{EllipticOperator}

%  \import{ZRandom/}{MatrixMult}

%  \import{ZRandom/}{SurfaceInt}
  
%  \import{ZRandom/}{Arrangement}

%  \import{ZRandom/}{ComplexRoot}

%  \import{ZRandom/}{SVD}
%
%  \import{ZRandom/}{ProbvsAnal}
  
%  \import{ZRandom/}{WeakDerivative}
  
%  \import{ZRandom/}{LinearFunctional}
%
%  \import{ZRandom/}{KolmogorovExist}
  
%  \import{ZRandom/}{OpenClosed}

%  \import{ZRandom/}{QuadraticForm}
  
%  \import{ZRandom/}{Divergence}
%  
%  \import{ZRandom/}{Mollifier}
%
%  \import{ZRandom/}{Tensors}
%
%  \import{ZRandom/}{Permutations}
%
%  \import{ZRandom/}{Christoffel}
%
%  \import{ZRandom/}{ChainRule}
%
%  \import{ZRandom/}{StochInt}
%
%  \import{ZRandom/}{Fatou}
%
%  \import{ZRandom/}{Kolmogorov01}
%
%  \import{ZRandom/}{Hewitt}
%
%  \import{ZRandom/}{Divisibility}
%
%  \import{ZRandom/}{BM}
%
%  \import{ZRandom/}{GreenFunction}
%
%  \import{ZRandom/}{Fourier}
%
%  \import{ZRandom/}{Similar}
%
%  \import{ZRandom/}{SobolevEmbedding}
%
%  \import{ZRandom/}{OlymInequalities}
%
%  \import{ZRandom/}{DetArea}

%  \import{ZRandom/}{Poisson}
%
%  \import{ZRandom/}{Excursions}
%
%  \import{ZRandom/}{LocalTimeHittingTime}
  
%  \import{ZRandom/}{Equality}
%
%  \import{ZRandom/}{MarkovProcess}
%  
%  \import{ZRandom/}{Operator}
%  
%  \import{ZRandom/}{Sequences}
%
%  \import{ZRandom/}{Minkowski}
%
%  \import{ZRandom/}{HolderIneq}
%
%  \import{ZRandom/}{MaxMinOperations}

%  \import{ZRandom/}{SpectralTheorem}

%  \import{ZRandom/}{QuantumOperators}

%  \import{ZRandom/}{Schrodinger}

%  \import{ZRandom/}{PowerRule}

  \import{ZRandom/}{Laplace}

%  \import{ZRandom/}{LagrangeMultiplier}



%  \import{ZRandom/}{Predator}

  



%------------------------PAPER SUMMARY------------------------

%  \import{PaperSummary/}{ZChapter}

    

%----------------------------------THESIS-------------------------------------

%  \import{Thesis/}{Going}
%  \import{Thesis/}{ThesisSummary}







%==================PHYSICS========================

%  \import{Physics/Mechanics/Goldstein}{Ch1}
%  \import{Physics/Mechanics/Goldstein}{Ch2}  
%  \import{Physics/Mechanics/Goldstein}{Ch5}    
%  \import{Physics/Mechanics/Goldstein}{Ch7}
%  \import{Physics/Mechanics/Goldstein}{Ch8}  
%  \import{Physics/Mechanics/Goldstein}{Ch9}        
%  \import{Physics/Mechanics/Goldstein}{Ch10}

%  \import{Physics/Relativity/Hartle}{Ch3}    
%  \import{Physics/Relativity/Hartle}{Ch4}    
        
%  \import{Physics/EM/Griffith}{Ch1}
%  \import{Physics/EM/Griffith}{Ch2}

%  \import{Physics/EM/Garrity}{Ch2}
%  \import{Physics/EM/Garrity}{Ch3}
%  \import{Physics/EM/Garrity}{Ch4}
%  \import{Physics/EM/Garrity}{Ch7}
%  \import{Physics/EM/Garrity}{Ch8}






%------------------------OBJECTS------------------------

%  \include{ZObjects}
%  \input{ZScratch.tex}    



    
\end{onehalfspacing}
\end{document}


